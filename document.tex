\documentclass{article}
\usepackage[T1]{fontenc}
\usepackage[utf8]{inputenc}
\usepackage{lmodern}
\usepackage[a4paper]{geometry}
\usepackage{babel}
\userpackage{titlesec}

\usepackage{graphicx}
\begin{document}
	voici un simple exemple \textbf{de mise en gras avec}\LaTeX{} la mise en italique se fait avec \textit{ce texte est en italic} quant est il de la mise en capital : on utilise la commande \textsc{pour mettre le texte en majuscule} pour les familles type machine on utilise la commande \texttt{qui permet la mise en forme des familles type machine } ou \textsf{empattement}.\\
	
		Bon pour repartir a la ligne il faut juste taper sur la touche entrer et voila c'est bon \textbf{j'espere que vous avez remarquer
    l'echappement pour la mise en des caracteres '} pour les accents, on utilise l'echappement \textsf{ \large \'e , \^ , \c{c})} de meme on utiluse la commande {\tiny pour les texte en miniscule} pour {\huge pour les textes tres enorme}\\
		\title{les environements : }
	On utilise les environements pour creation des objets evolues notamment les \textbf{les tableaux, les listes ordonn\'ees et non ordonn\'ees , les listes de definition, la mise des fontes, les taille des texte ..} on utilise de se fait la syntasce begin{} puis le nom de la commande a appliquer.\\
	Exple de listes :
	\begin{itemize}
		
		\item \textbf{itemize} pour definir les listes
		\item \textbf{enumerate(1., 2., ...) }pour les enumerations
		\item \textbf{description} pour les listes de description
		\item \textbf{tabular} pour la creation de tableau
	\end{itemize}		 
	On essaie maintenant la commande d'environement de centrage
	\begin{center}
		\begin{verbatim}
			ce texte doit \^etre simplement centre et il est dans la variable d'environement verbatim 
		\end{verbatim}
ce texte n'etait pas r\'eellement dans l'environement \textbf{center};
	\end{center}

commen\c{c}ont maintenant $\acute{a}$ la ligne

	\begin{tabular}{|r|l|l|c|}
		\hline
		pays & & capitale & contiment\\
		\hline
		Burkina Faso & & Ouagadougou & Afrique\\
		\hline
		France & & Paris & Europe\\
		\hline
		Rwanda & & Kigali & Afrique\\
		\hline
 & & & \\
 \hline
	\end{tabular}
\begin{center}
	\textbf{Creation de nouveau tableau comme exercice}
\end{center}
\textbf{le sectionnement de mon document} \\
Le sectionnement permet une bonne disposition des paragraphes et une structuration correcte de notre document pour se faire on utilise les commande dites de sectionnement que sont: chaper{}, paragraphe{}, section{} avec leur variant \'etoil\'e comme section*{}
\\
\underline{Exple: }\\ 
\section{Introduction}
\label{intro}
\section{le cinema}
le cinema est une activit\'e pleinement importante dans la vie des \^etre humans 
\subsection{Ces Acteurs}
\begin{itemize}
	\item Realisateur
	\item cameraman
	\item les acteurs
	\item les cine\^{a}tre 
\end{itemize}

\subsection{les adeptes du cinema }
\section{La musique}
d'ici je souhaite referencer l'introduction ~\ref{intro} va
\section*{Conclusion}

\tableofcontents{table de matiere}
\footnotemark{mark}
\footnotetext{ceci est mise comme note de page
	}\paragraph{Etude du march\'e}
% TODO: \usepackage{graphicx} required
\begin{center}
	\includegraphics[width=10cm, height=1cm]{../../Images/pupy}
\end{center}
\begin{figure}[ht]
	\centering
	\includegraphics[ height=1cm]{../../Images/pupy}\\
	bla bla (commentaite sur la figure)
	\caption{une caption sur l'image}
	\label{key}
	
\end{figure}
\begin{frame}
	slide
\end{frame}


\vspace*{\stretch{1}}
\begin{center}
	\begin{minipage}{10cm}
		
	\end{minipage}
\end{center}
\vspace*{\stretch{1}}


\end{document}
